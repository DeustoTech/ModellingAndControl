\chapter{Crop Model}

    El crecimiento de las plantas de tomate esta determinado por las variables climáticas: radiación, temperatura. Estas variables determinan los procesos biológicos dentro de la planta, como por ejemplo: 

    \begin{itemize}
        \item Photosynthesis. Proceso de transformación de radiación solar a carbohidratos.
        \item Photorespiration. Consumo de carbohidratos para el mantenimiento de procesos biológicos.
    \end{itemize}

    Aunque el crecimiento de la planta tiene depdnedencias con respecto a los nitrientes aportados o la humédad relativa, siempre que de la cosecha no tenga ni limitación de agua, ni limitación de nutrientes estas variables pueden no ser consideradas. 

    Tomaremos como referencia el modelo \emph{HortSys} \cite{Martinez-Ruiz2019} este  es un modelo simplificado de producción de tomates en invernaderos con sistemas de hidropónia. Hemos agregado de una variable más que representa la temperatura media cada veinticuatro horas y hemos cambiado la formulación de tiempo discreto a tiempo continuo.


    \section{Variables Description}

    Esta sección esta dedicada a la descripción de cada un de las variables involucradas en el modelo dinámico, con el fin de asimilar su significado físico. A continuación describiremos en detalle cada una de las variables necesarias
    \subsection{Input Variables} 
    Consideraremos que somos capaces de medir de manera instantánea los valores de temperatura y radiación recibida por la plantación. 
    \begin{enumerate}
        \item \textbf{Global Radiation } ($u_R[W/m^2]$) Radiación recibida entre el area total del invernadero.
        \item \textbf{Temperature} ($u_T[^\circ C]$) Temperatura en el interior del invernadero.
    \end{enumerate}


    \subsection{State Variables} 
    En HortSys las variables de estado guardan la información sobre la radiación acumulada y la temperatura durante todo el día. A partir de esta variables se pueden realizar estimaciones en concreto de LAI. 
        
    \begin{enumerate}

        \item \textbf{Photothermal time} (pt): Denotada por $x_{pt}[Js^{-1}]$ representa la energía electromagnética capturada en un rango de temperatura concreto. Este rango de temperatura actua como una ventantada de temperatura donde la captura de radiación es posible. Esta variable determina el crecimiento de las cada uno de los distintos componentes de la planta, además de marcar las etapas de crecimiento.

        
        \item \textbf{Dry Fruit Matter}(dfm): Denotada por $x_{dfm}[kg]$, es la  cantidad de tomate en la cosecha. Deberemos diferenciar el tomate que ha sido recogido, con el tomate verde que se encuentra en la planta. Por lo general los modelos se refieren a los kilos de tomate recogido, sin embargo en modelos propuestos en \cite{VanthoorThesis} o 
 
        \item \textbf{Dry Steam Matter} ($x_{dfm}[kg]$): \emph{dsm} Masa de carbohidratos correspondientes a la de ramas.
        
        \item \textbf{Dry Leaf Matter} ($x_{dlm}[kg]$): \emph{dlm} Masa de hojas de toda la plantación.
    
        \item \textbf{Dry Matter Production} ($x_{dmp}[kg]$): \emph{dmp} es una medida de la cantidad de masa en la totalidad de la planta. Es decir, esta contiene a la masa total de los frutos, hojas, ramas y raices. Esta variables es una simplificación que aglutina la información de $x_{dsm}$, $x_{dfm}$ y $x_{dlm}$.
     
    \end{enumerate}

    
    \subsection{Measurements Variables}
    En \cite{Martinez-Ruiz2019} se describen varias variables intermedias que tienen significado físico. Sin embargo en aras de la simplicidad de las expresiones, denotaremos estas variables intermedias como la letra $y$. Es importante notar que estas variables son dependientes de las variables de estado $x$, por lo que en cada 

    \begin{enumerate}
        \item \textbf{Leaf Area Index} ($y_{la}[(m^2 hojas)/(m^{2} invernadero )]$): Se define como Leaf Area Index al area total de hojas de total de la plantación entre el area total de cultivo. Esta varaible  nos da un medida de las dimensión de la plantación. Suele tomar valores del orden de $(1 \sim 10) \ m^2/m^2$  
    
        \item \textbf{Intercepted Radiation Fraction} ($y_{fpar}[-]$): Es la fracción radiación que puede aprovechar la plantación conrespecto a la radiación total recibida en todo el invernadero. Por su naturaleza de fracción solo puede tomar valores comprendidos en $y_{fpar} \in [0,1]$.
        \item \textbf{Photosynthetical Active Radiation} ($y_{PAR}[Wm^{-2}]$): Esta es la fracción de radiación de todo el espectro electromagnético que contribuye a la fotosíntesis. 

    
    \end{enumerate} 