
    \chapter{Data Pre-procesing }

    \section{Test and Train Trayectories }


\begin{dataset}[without Heater]\label{dataset:noheater}
   Abarca desde 03-Feb-2019 hasta 15-Feb-2019. En esta trayectoria del sistema no se observa oscilaciones en la horas de mínima temperatura. De manera que, supondremos que durante esta trayectoria la caldera siempre se encuentra apagada.
\end{dataset}

\begin{dataset}[Heater]\label{dataset:heater}
    Abarca desde 17-Feb-2019 hasta 28-Feb-2019. En esta trayectoria del sistema no se observa oscilaciones en la horas de mínima temperatura. Es decir, podemos presuponer que no existe calor aportado por la caldera de pellet en esta base de datos
\end{dataset}

De manera que, supondremos que durante esta trayectoria la caldera se comporta como se esta descrito en el deliverable D1.1:
\begin{verse}\textbf{From D1.1 Page 82} 
    Each pellet boiler has a PLC for its automatic activation and/or deactivation based on the value of air temperature inside the greenhouse (measured with an integrated PT100 temperature probe). The temperature set point is usually set at $12 ^\circ C$.
\end{verse}

\section{Obtención de Señal de clima con la mínima composición harmónica}


\section{Detección Automática de calor desde la cladera de pellet}

\section{Selección de sesiones de cultivo}
